\documentclass[a4paper]{jpconf}
\usepackage{graphicx}
\usepackage{amsmath}
\begin{document}
\title{Neutron detection in the SNO+ water phase}

\author{Y Liu, on behalf of the SNO+ collaboration}
%\author{Y Liu, S Andringa, D Auty, F Bar\tilde{a}o}

\address{Department of Physics, Engineering Physics \& Astronomy, Queen's University, ON, Canada}

\ead{yan.liu@owl.phy.queensu.ca}

\begin{abstract}
SNO+ is a multipurpose neutrino experiment located approximately 2 km underground in SNOLAB, Sudbury, Canada. The detector started taking physics data in May 2017 and is currently completing its first phase, as a pure water Cherenkov detector. The low trigger threshold of the SNO+ detector allows for a substantial detection efficiency of these neutrons, as observed with a deployed $^{241}$Am$^{9}$Be source. Using a statistical analysis of a one hour AmBe calibration data, we report a neutron capture constant of 208.2$\pm$2.1(\textit{stats.}) $\mu$s and a lower bound of the neutron detection efficiency of 46\% in the detector center.
\end{abstract}

\section{Introduction}

The SNO+ experiment\cite{Andringa:2015tza} is an underground multipurpose neutrino experiment that reuses the infrastructure of the SNO experiment. Located $\sim$2000 meters underground, The detector consists of a 12 meter diameter spherical acrylic vessel (AV), which is currently filled with $\sim$1000 tonnes ultra pure water. Outside the AV, $\sim$9400 PMTs are mounted on a 18 meter diameter geodesic stainless steel structure (PSUP). During the water phase, the experiment explores several interesting aspects of particle physics, including possible antineutrino searches using neutron tagging.

Electron antineutrinos are measured in most experiments via the inverse beta decay, as it provides a delayed neutron signal which can be used to suppress random background. In SNO+, neutrons propagating in light water will thermalize and then get captured by a proton, giving out a 2.2 MeV $\gamma$. Due to the small amount of energy deposited in the detector, neutron capture signals are very hard to detect in traditional pure water Cherekov detectors. One approach\cite{Watanabe:2008ru} was explored in a previous experiment, which involves a forced trigger system. Thanks to the good Cherekov light yield and upgraded electronics in SNO+, we are able to lower the detector threshold to $\sim$1 MeV, and therefore can trigger on the 2.2 MeV $\gamma$s with substantial efficiency. In this proceeding we present the first result of neutron detection in a light water Cherekov detector with normal trigger settings. 


\section{AmBe calibration}


$^{241}$Am$^{9}$Be source is a common neutron source that mimics the electron antineutrino signal very well. The $\alpha$-particle emitted by $^{241}$Am (half-life of 432.2 years) can be absorbed by the $^{9}$Be target, which will then decay into $^{12}$C through neutron emission. About 60\% of the time $^{12}$C is produced in an excited state, and the immediate de-excitation predominantly emits a 4.4 MeV $\gamma$. The neutron, on the other hand, will thermalize and then get captured by a proton in $\sim$200 $\mu$s. The source provide an coincident signal with the \textit{prompt} event being the 4.4 MeV $\gamma$ and the \textit{delayed} event being the neutron capture.

A SNO AmBe source was identified with a nominal strength of 1683.33 kBq and a neutron rate of 62 Hz\cite{jloach2009:diss}. The source was doubly encapsulated with black delrin. However, because the source has been used by other experiments, the cleanliness of the encapsulation was not clear. Therefore, a new encapsulation was designed and fabricated to further reduce the risk of detector contamination. The new encapsulation was leaked tested and thoroughly cleaned prior to the source deployment.

SNO+ reuses most of the SNO infrastructure for calibration\cite{Moffat:2005tq} in the Water phase with essential upgrades on the side rope boxes. The Manipulator system, show schematically in Figure \ref{fig:1}, contains a Umbilical Retrieval Mechanism (URM) which is mounted on the Universal Interface (UI). Calibration sources can therefore be put in different positions inside the detector, when attched to a 100 ft Umbilical and side ropes. 

\begin{figure}[h]
\includegraphics[width=14pc]{figures/manip.pdf}\hspace{2pc}%
\begin{minipage}[b]{14pc}\caption{\label{fig:1}A schematical drawing of the Manipulator system in SNO+.}
\end{minipage}
\end{figure}

We deployed the AmBe source in 15 different positions along the two axes: the vertical z-axis and the y-axis which points to the North. This allowed detailed studies across the full detector volume. The total AmBe calibration time is $\sim$ 16 hours.

\section{AmBe data analysis}

Because the AmBe source is an untagged source, traditional analyses put stringent time, position as well as other cuts to obtain a relatively pure sample. This method requires estimation of the background contamination, which often leads to large systematic uncertainties. The statistics can also be much reduced because of the stringent cuts. Here we present a statistical analysis that avoids the above problems, and also gives a direct measurement of the neutron detection efficiency.

The analysis started by filling a histogram with the time differences between the \textit{prompt} event and the \textit{delayed} event. These \textit{Prompt} events and \textit{delayed} events are selected with a minimum Nhits (number of fired PMTs in one event) cut. Three components can be categorized in this histogram:

\begin{itemize}
\item True-True event: the \textit{prompt} is the 4.4 MeV $\gamma$, and the \textit{delayed} is the associated neutron. 
\item True-Fake event: the \textit{prompt} is the 4.4 MeV $\gamma$, but the \textit{delayed} is background event.
\item Fake-Fake event: both \textit{prompt} and \textit{delayed} are backgrounds. The distribution of Fake-Fake events will follow an exponential with an exponential constant of the background rate.
\end{itemize}

For True-Fake event, if the neutron does not trigger the detector or get removed by the Nhits cut, the distribution will be an exponential with an exponential constant of the background rate. However, if the neutron triggers the detector and passes the Nhits cut, the following \textit{delayed} event can be either the associated neutron, or a background event that happens to appear before the neutron. Therefore, the probability of the \textit{delayed} event being a background is:
\begin{equation}
\begin{aligned}
\textrm{Prob}_{\gamma - b\textrm{ before N }}(\textrm{t}) & = P \cdot E \cdot R_{2} e^{-R_{2}\textrm{t}} \cdot (1- \int_0^{\textrm{t}} \lambda e^{-\lambda \textrm{t'}}\textrm{dt'}) \\
& = P \cdot E \cdot R_{2} e^{-(R_{2}+\lambda)\textrm{t}}
\end{aligned}
\end{equation}

where $P$ is the fraction of true 4.4 MeV $\gamma$s in the \textit{prompt} events, $E$ is the neutron detection efficiency, $\lambda$ is the neutron capture constant and $R_2$ is the background rate.

Similarly, for True-True event we have:
\begin{equation}
\textrm{Prob}_{\gamma - N\textrm{ before b }}(\textrm{t}) = P \cdot E \cdot \lambda e^{-(\lambda+R_{2})\textrm{t}}
\end{equation}

Therefore, we derived the fit function for the time difference histogram:

\begin{equation}
F(\textrm{t}) = N \cdot R_{1} (P \cdot E \cdot (\lambda+R_{2}) e^{-(\lambda+R_{2}) \textrm{t}} + (1-P \cdot E) \cdot R_{2} e^{-R_{2} \textrm{t}})
\end{equation}

where $N$ is a normalization factor associated with detector livetime and histogram bin size and $R_{1}$ is the \textit{prompt} event rate. Figure \ref{fig:2} shows the three different components of the time difference histogram from a toy MC model.

\begin{figure}[h]
\begin{center}
\includegraphics[scale=0.5]{./figures/toymc_ambe2.pdf}
\end{center}
\caption{\label{fig:2}Components of the time difference histogram from a toy MC model.}
\end{figure}

\section{Analysis results}

We present here the analysis result for a one hour AmBe run where the source is placed at the center of the detector. Figure \ref{fig:3} shows the fitted results. We report a neutron capture constant of 208.2$\pm$2.1(\textit{stats.}) $\mu$s, which is consist with previous measurements\cite{Super-Kamiokande:2015xra}\cite{Cokinos:1977zz}.

\begin{figure}[h]
\begin{center}
\includegraphics[scale=0.5]{./figures/TimeDifference_log.pdf}
\end{center}
\caption{\label{fig:3}Time difference histogram fitted with the analytical function. Shown here are data from a one hour central run.}
\end{figure}

By varying the Nhits cut on the \textit{prompt} events, we found the maximum $P \cdot E$ value to be 46\%. By definition $P$ is smaller than 1,therefore we derived a conservative lower limit for $E$ at the center of the detector:

\begin{equation}
E > (P \cdot E)_{max} = 46\%
\end{equation}

This is the highest neutron detection efficiency achieved to date in a pure water Cherekov detector.

Notice that $R_1 \cdot P \cdot E$ is the rate of true $\gamma$-n coincidences, regardless of whether there is a background event before the neutron. By calculating the difference of $R_1 \cdot P \cdot E$ for two consecutive Nhits cut, we can plot the Nhits distributions for both the 4.4 MeV $\gamma$s and the neutrons, which is shown in Figure \ref{fig:4}. Note that using this method, the Nhits distributions is not biased by random background and therefore can be used for energy calibration in SNO+. This not only provides an addtional check on the energy scale linearity, but also helps to constrain measurement of backgrounds from U/Th chains. 

\begin{figure}[h]
\begin{center}
\includegraphics[scale=0.5]{./figures/Nhits_with_total_rate.pdf}
\end{center}
\caption{\label{fig:4}Derived Nhits distributions for both the 4.4 MeV $\gamma$s and the neutrons. The Nhits distributions are compared with the total event rate in this AmBe run. The Nhits distributions agree well with MC.}
\end{figure}

\section{Conclusion}

SNO+ started its water phase in May 2017 and has been steadily taking data since then. An AmBe source was deployed in the SNO+ detector for energy calibration and potential antineutrino searches. In this proceeding we presented a novel data analysis method. With one hour central run data, we measured the neutron capture constant to be 208.2$\pm$2.1(\textit{stats.}) $\mu$s. A lower limit of 46\% neutron detection efficiency was obtained at the SNO+ detector center. This is the highest neturon detection efficiency achieved to date in a pure water Cherekov detector. 


\ack{}

This work is supported by ASRIP, CIFAR, CFI, DF, DOE, ERC, FCT, FedNor, NSERC, NSF, Ontario MRI, Queen's University, STFC, UC Berkeley and benefitted from services provided by EGI, GridPP and Compute Canada. The author thanks FCT (Funda\c{c}$\tilde{a}$o para a Ci$\hat{e}$ncia e a Tecnologia, Portugal) and the Arthur B. McDonald Canadian Astroparticle Physics Research Institute for financial support. We thank SNOLAB and Vale for valuable support.

\section*{References}
\bibliographystyle{iopart-num}
\bibliography{ref}



\end{document}
