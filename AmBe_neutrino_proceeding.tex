\documentclass[a4paper]{jpconf}
\usepackage{graphicx}
\usepackage{amsmath}
\begin{document}
\title{Neutron detection in the SNO+ water phase}

\author{Yan Liu, on behalf of the SNO+ collaboration}

\address{Department of Physics, Engineering Physics \& Astronomy, Queen's University, ON, Canada}

\ead{yan.liu@owl.phy.queensu.ca}

\begin{abstract}
SNO+ is a multipurpose neutrino experiment located approximately 2 km underground in SNOLAB, Sudbury, Canada. The detector started taking physics data in May 2017 and is currently completing its first phase, as a pure water Cherenkov detector. One physics goal of this water phase is to detect reactor antineutrinos for the first time using a pure water target. A key component of this search is the identification of the 2.2 MeV $\gamma$ from the inverse beta decay neutron capturing on hydrogen. The low trigger threshold of the SNO+ detector allows for a substantial detection efficiency of these neutrons, as observed with a deployed $^{241}$Am$^{9}$Be source. This poster presents the recent effort on the AmBe calibration and an analysis of the obtained coincident pairs of 4.4 MeV and 2.2 MeV $\gamma$s in the SNO+ detector.
\end{abstract}

\section{Introduction}

The SNO+ experiment \cite{iopartnum} is a deep underground neutrino experiment that reuses the infrastructure of the SNO experiment. Located $\sim$2000 meters underground, The detector consists of a 12 meter diameter spherical acrylic vessel (AV), which is currently filled with $\sim$1000 tonnes ultra pure water. $\sim$9400 PMTs are mounted on a geodesic stainless steel structure (PSUP) of $\sim$8.9m radius. The detector is currently finished its first phase, the Water phase, during which period the experiment explores several interesting aspects of particle physics, including antineutrinos. $FIXME$ In this proceeding, neutron detection will be presented.


Neutrons propagating in light water will thermalize and then get captured by a proton, giving out a 2.2 MeV $\gamma$. Due to the small amount of energy deposited in the detector by such $\gamma$, the neutron capture signal is very hard to detect in traditional Water Cherekov detectors. Two approaches \cite{SK} have been explored, which involves either loading extra chemicals or introducing forced trigger system. Thanks to the good Cherekov light yield and upgraded electronics in SNO+, we are able to trigger on 2.2 MeV $\gamma$s with reasonable efficiency. In this proceeding we present the first result of neutron detection in a light water Cherekov detector with normal trigger settings. 


\section{AmBe calibration}


The $\alpha$-particle emitted by $^{241}$Am (half-life of 432.2 years) can be absorbed by the $^{9}$Be target, which will then decay into $^{12}$C through neutron emission. About 60\% of the time $^{12}$C is produced in an excited state, and de-excitation predominantly emits a 4.4 MeV $\gamma$. The neutron, on the other hand, will thermalize and then get captured by a proton, giving out a 2.2 MeV $\gamma$.

An old SNO experiment AmBe source was identified with a nominal strength of 1683.33 kBq and a neutron rate of 62 Hz.\cite{} The source came doubly encapsulated with black delrin. However, because the source has been used by other experiments, the cleanliness is not clear to SNO+. Therefore, a new encapsulation was designed and fabricated to reduce the risk of contaminating the detector to minimal. Below is a drawing of the new encapsulation. The new encapsulation has passed pressure test and bubble test. A detailed cleaning procedure was performed prior to the source entry into the detector.

SNO+ reuses most of the SNO infrastructure for calibration in the Water phase.\cite{} The Manipulator system, show schematically in Figure \ref{fig1}, contains a Umbilical Retrieval Mechanism (URM) which is mounted on the Universal Interface (UI). Different calibration sources can therefore be put in different positions inside the detector, when attched to a 100 ft Umbilical and side ropes. 

We deployed the AmBe source in 15 different positions along the two axes: the vertical z-axis and the y-axis which points to the North. This allows detailed studies across the full detector volume. The total AmBe calibration time is around 16 hours.

\section{AmBe data analysis}

Because the AmBe source is an untagged source, traditional analyses put stringent time, position and other cuts to achieve a relatively pure sample. These methods requires estimate of background contamination, which tends to have large systematic uncertainties. The statistics can also be much reduced because of the stringent cuts. Here we present a statistical analysis that avoids the above problems, and also gives a direct measurement of the neutron detection efficiency.

The analysis started by filling a histogram with the time differences between the \textit{prompt} event and the \textit{delayed} event. These \textit{Prompt} events and \textit{delayed} events are selected with a minimum nhits (number of fired PMTs in one event) cut. Three components can be categorized in this histogram:

\begin{itemize}
\item Fake-Fake event: both \textit{prompt} and \textit{delayed} are backgrounds. The distribution of Fake-Fake events will follow an exponential with a constant of the background rate.
\item True-Fake event: the \textit{prompt} is the 4.4 MeV $\gamma$, but the \textit{delayed} is background event.
\item True-True event: the \textit{prompt} is the 4.4 MeV $\gamma$, and the \textit{delayed} is the associated neutron. 
\end{itemize}

For True-Fake event, if the neutron does not trigger the detector or got removed by the nhits cut, the distribution will be an exponential with a constant of the background rate. However, if the neutron triggers the detector and passed the nhits cut, the following \textit{delayed} event can be either the associated neutron, or a background event but happened to appear before the neutron. Therefore, the probability of the \textit{delayed} event being a background is:
\begin{equation}
\begin{aligned}
\textrm{Prob}_{\gamma - b\textrm{ before N }}(\textrm{t}) & = P \cdot E \cdot R_{2} e^{-R_{2}\textrm{t}} \cdot 1- \int_0^{\textrm{t}} \lambda e^{-\lambda \textrm{t'}}\textrm{dt'} \\
& = P \cdot E \cdot R_{2} e^{-(R_{2}+\lambda)\textrm{t}}
\end{aligned}
\end{equation}

where $P$ is the fraction of true 4.4 MeV $\gamma$s in the \textit{prompt} events, $E$ is the probability of detecting a neutron (neutron detection efficiency), $\lambda$ is the neutron capture constant and $R_2$ is the background rate.

Similarly, for True-True event we have:
\begin{equation}
\textrm{Prob}_{\gamma - N\textrm{ before b }}(\textrm{t}) = P \cdot E \cdot \lambda e^{-(R_{2}+\lambda)\textrm{t}}
\end{equation}

Therefore, we derived the fit function for the time difference histogram:

\begin{equation}
F(\textrm{t}) = N \cdot R_{1} (P \cdot E \cdot (\lambda+R_{2}) e^{-(\lambda+R_{2}) \textrm{t}} + (1-P \cdot E) \cdot R_{2} e^{-R_{2} \textrm{t}})
\end{equation}

where $N$ is a normalization factor (livetime $\times$ bin size) and $R_{1}$ is the \textit{prompt} event rate. Figure \ref{fig2} shows the different components of the time difference histogram from a toy MC model.

\section{Analysis results}

We present here the analysis result for a one hour calibration run where the source is placed at the center of the detector. Figure \ref{fig:3} shows the fitted results. We report a neutron capture constant of 208.2$\pm$2.1(\textit{stats.}) $\mu$s, which is consist with previous measurements\cite{}\cite{}\cite{}.

By varying the nhits cut on the \textit{prompt} events, we found the maximum $P \cdot E$ value to be 46\%. By definition $P$ is smaller than 1,therefore we derived a conservative lower limit for $E$ at the center of the detector:

\begin{equation}
E > (P \cdot E)_{max} = 46\%
\end{equation}

This is the highest neutron detection efficiency achieved to date in a pure water Cherekov detector.

Notice that $R_1 \cdot P \cdot E$ is the rate of true $\gamma$-n coincidences, regardless of whether there is a background event before the neutron. By calculating the difference of $R_1 \cdot P \cdot E$ for two consecutive nhits cut, we can plot the Nhits distribution for both the 4.4 MeV $\gamma$s and the neutrons, which is shown in Figure \ref{fig:4}. Note that with this method, the nhits distributions is not biased by background. Energy calibration using AmBe source is ongoing in SNO+.

\section{Conclusion}

SNO+ started its water phase in May 2017 and has been steadily taking data since then. An AmBe source was deployed in the SNO+ detector for energy calibration and potential antineutrino searches. In this proceeding we presented a novel data analysis method. With one hour central run data, we measured the neutron capture constant to be 208.2$\pm$2.1(\textit{stats.}). A lower limit of 46\% neutron detection efficiency was obtained at the SNO+ detector center. This is the highest neturon detection efficiency achieved to date in a pure water Cherekov detector. 


\ack{}

This work is supported by ASRIP, CIFAR, CFI, DF, DOE, ERC, FCT, FedNor, NSERC, NSF, Ontario MRI, Queen's University, STFC, UC Berkeley and benefitted from services provided by EGI, GridPP and Compute Canada. The author thanks FCT (Funda\c{c}$\tilde{a}$o para a Ci$\hat{e}$ncia e a Tecnologia, Portugal) and the Arthur B. McDonald Canadian Astroparticle Physics Research Institure for financial support. We thank SNOLAB and Vale for valuable support.

\iffalse
\section{The text}
The text of the article should should be produced using standard \LaTeX\ formatting. Articles may be divided into sections and subsections, but the length limit provided by the \corg\ should be adhered to.

\subsection{Acknowledgments}
Authors wishing to acknowledge assistance or encouragement from 
colleagues, special work by technical staff or financial support from 
organizations should do so in an unnumbered Acknowledgments section 
immediately following the last numbered section of the paper. The 
command \verb"\ack" sets the acknowledgments heading as an unnumbered
section.

\subsection{Appendices}
Technical detail that it is necessary to include, but that interrupts 
the flow of the article, may be consigned to an appendix. 
Any appendices should be included at the end of the main text of the paper, after the acknowledgments section (if any) but before the reference list.
If there are two or more appendices they will be called Appendix A, Appendix B, etc. 
Numbered equations will be in the form (A.1), (A.2), etc,
figures will appear as figure A1, figure B1, etc and tables as table A1,
table B1, etc.

The command \verb"\appendix" is used to signify the start of the
appendixes. Thereafter \verb"\section", \verb"\subsection", etc, will 
give headings appropriate for an appendix. To obtain a simple heading of 
`Appendix' use the code \verb"\section*{Appendix}". If it contains
numbered equations, figures or tables the command \verb"\appendix" should
precede it and \verb"\setcounter{section}{1}" must follow it. 

\section{References}
%%%%%%%%%%%%%%%%%%%%%%%%%%%%%%%%%%%%%%%%%%%
In the online version of \jpcs\ references will be linked to their original source or to the article within a secondary service such as INSPEC or ChemPort wherever possible. To facilitate this linking extra care should be taken when preparing reference lists. 

Two different styles of referencing are in common use: the Harvard alphabetical system and the Vancouver numerical system.  For \jpcs, the Vancouver numerical system is preferred but authors should use the Harvard alphabetical system if they wish to do so. In the numerical system references are numbered sequentially throughout the text within square brackets, like this [2], and one number can be used to designate several references.  

\subsection{Using \BibTeX}
We highly recommend the {\ttfamily\textbf\selectfont iopart-num} \BibTeX\ package by Mark~A~Caprio \cite{iopartnum}, which is included with this documentation.

\subsection{Reference lists}
A complete reference should provide the reader with enough information to locate the article concerned, whether published in print or electronic form, and should, depending on the type of reference, consist of:  

\begin{itemize}
\item name(s) and initials;
\item date published;
\item title of journal, book or other publication; 
\item titles of journal articles may also be included (optional);
\item volume number;
\item editors, if any;
\item town of publication and publisher in parentheses for {\it books};
\item the page numbers.
\end{itemize}

Up to ten authors may be given in a particular reference; where 
there are more than ten only the first should be given followed by 
`{\it et al}'. If an author is unsure of a particular journal's abbreviated title it is best to leave the title in 
full. The terms {\it loc.\ cit.\ }and {\it ibid.\ }should not be used. 
Unpublished conferences and reports should generally not be included 
in the reference list and articles in the course of publication should 
be entered only if the journal of publication is known. 
A thesis submitted for a higher degree may be included 
in the reference list if it has not been superseded by a published 
paper and is available through a library; sufficient information 
should be given for it to be traced readily.

\subsection{Formatting reference lists}
Numeric reference lists should contain the references within an unnumbered section (such as \verb"\section*{References}"). The 
reference list itself is started by the code 
\verb"\begin{thebibliography}{<num>}", where \verb"<num>" is the largest
number in the reference list and is completed by
\verb"\end{thebibliography}". 
Each reference starts with \verb"\bibitem{<label>}", where `label' is the label used for cross-referencing. Each \verb"\bibitem" should only contain a reference to a single article (or a single article and a preprint reference to the same article).  When one number actually covers a group of two or more references to different articles, \verb"\nonum"
should replace \verb"\bibitem{<label>}" at
the start of each reference in the group after the first.

For an alphabetic reference list use \verb"\begin{thereferences}" ... \verb"\end{thereferences}" instead of the
`thebibliography' environment and each reference can be start with just \verb"\item" instead of \verb"\bibitem{label}"
as cross referencing is less useful for alphabetic references.

\subsection {References to printed journal articles}
A normal reference to a journal article contains three changes of font (see table \ref{jfonts}) and is constructed as follows:

\begin{itemize}
\item the authors should be in the form surname (with only the first letter capitalized) followed by the initials with no periods after the initials. Authors should be separated by a comma except for the last two which should be separated by `and' with no comma preceding it;
\item the article title (if given) should be in lower case letters, except for an initial capital, and should follow the date;
\item the journal title is in italic and is abbreviated. If a journal has several parts denoted by different letters the part letter should be inserted after the journal in Roman type, e.g. {\it Phys. Rev.} A;
\item the volume number should be in bold type;
\item both the initial and final page numbers should be given where possible. The final page number should be in the shortest possible form and separated from the initial page number by an en rule `-- ', e.g. 1203--14, i.e. the numbers `12' are not repeated.
\end{itemize}

A typical (numerical) reference list might begin

\medskip
\begin{thebibliography}{9}
\item Strite S and Morkoc H 1992 {\it J. Vac. Sci. Technol.} B {\bf 10} 1237 
\item Jain S C, Willander M, Narayan J and van Overstraeten R 2000 
{\it J. Appl. Phys}. {\bf 87} 965 
\item Nakamura S, Senoh M, Nagahama S, Iwase N, Yamada T, Matsushita T, Kiyoku H 
and 	Sugimoto Y 1996 {\it Japan. J. Appl. Phys.} {\bf 35} L74 
\item Akasaki I, Sota S, Sakai H, Tanaka T, Koike M and Amano H 1996 
{\it Electron. Lett.} {\bf 32} 1105 
\item O'Leary S K, Foutz B E, Shur M S, Bhapkar U V and Eastman L F 1998 
{\it J. Appl. Phys.} {\bf 83} 826 
\item Jenkins D W and Dow J D 1989 {\it Phys. Rev.} B {\bf 39} 3317 
\end{thebibliography}
\smallskip

\noindent which would be obtained by typing

\begin{verbatim}
\begin{\thebibliography}{9}
\item Strite S and Morkoc H 1992 {\it J. Vac. Sci. Technol.} B {\bf 10} 1237 
\item Jain S C, Willander M, Narayan J and van Overstraeten R 2000 
{\it J. Appl. Phys}. {\bf 87} 965 
\item Nakamura S, Senoh M, Nagahama S, Iwase N, Yamada T, Matsushita T, Kiyoku H 
and 	Sugimoto Y 1996 {\it Japan. J. Appl. Phys.} {\bf 35} L74 
\item Akasaki I, Sota S, Sakai H, Tanaka T, Koike M and Amano H 1996 
{\it Electron. Lett.} {\bf 32} 1105 
\item O'Leary S K, Foutz B E, Shur M S, Bhapkar U V and Eastman L F 1998 
{\it J. Appl. Phys.} {\bf 83} 826 
\item Jenkins D W and Dow J D 1989 {\it Phys. Rev.} B {\bf 39} 3317 
\end{\thebibliography}
\end{verbatim}

\begin{center}
\begin{table}[h]
\centering
\caption{\label{jfonts}Font styles for a reference to a journal article.} 
\begin{tabular}{@{}l*{15}{l}}
\br
Element&Style\\
\mr
Authors&Roman type\\
Date&Roman type\\
Article title (optional)&Roman type\\
Journal title&Italic type\\
Volume number&Bold type\\
Page numbers&Roman type\\
\br
\end{tabular}
\end{table}
\end{center}

\subsection{References to \jpcs\ articles}
Each conference proceeding published in \jpcs\ will be a separate {\it volume}; 
references should follow the style for conventional printed journals. For example:\vspace{6pt}
\numrefs{1}
\item Douglas G 2004 \textit{J. Phys.: Conf. Series} \textbf{1} 23--36
\endnumrefs

%%%%%%%%%%%%%%%%%%%%%%%%%%%%%%%%%%
\subsection{References to preprints}
For preprints there are two distinct cases:
\renewcommand{\theenumi}{\arabic{enumi}}
\begin{enumerate}
\item Where the article has been published in a journal and the preprint is supplementary reference information. In this case it should be presented as:
\medskip
\numrefs{1}
\item Kunze K 2003 T-duality and Penrose limits of spatially homogeneous and inhomogeneous cosmologies {\it Phys. Rev.} D {\bf 68} 063517 ({\it Preprint} gr-qc/0303038)
\endnumrefs
\item Where the only reference available is the preprint. In this case it should be presented as
\medskip
\numrefs{1}
\item Milson R, Coley A, Pravda V and Pravdova A 2004 Alignment and algebraically special tensors {\it Preprint} gr-qc/0401010
\endnumrefs
\end{enumerate}

\subsection{References to electronic-only journals}
In general article numbers are given, and no page ranges, as most electronic-only journals start each article on page 1.

\begin{itemize} 
\item For {\it New Journal of Physics} (article number may have from one to three digits)
\numrefs{1}
\item Fischer R 2004 Bayesian group analysis of plasma-enhanced chemical vapour deposition data {\it New. J. Phys.} {\bf 6} 25 
\endnumrefs
\item For SISSA journals the volume is divided into monthly issues and these form part of the article number

\numrefs{2}
\item Horowitz G T and Maldacena J 2004 The black hole final state {\it J. High Energy Phys.}  	JHEP02(2004)008
\item Bentivegna E, Bonanno A and Reuter M 2004 Confronting the IR fixed point cosmology 	with 	high-redshift observations {\it J. Cosmol. Astropart. Phys.} JCAP01(2004)001  
\endnumrefs
\end{itemize} 

\subsection{References to books, conference proceedings and reports}
References to books, proceedings and reports are similar to journal references, but have 
only two changes of font (see table~\ref{book}). 

\begin{table}
\centering
\caption{\label{book}Font styles for references to books, conference proceedings and reports.}
\begin{tabular}{@{}l*{15}{l}}
\br
Element&Style\\
\mr
Authors&Roman type\\
Date&Roman type\\
Book title (optional)&Italic type\\
Editors&Roman type\\
Place (city, town etc) of publication&Roman type\\
Publisher&Roman type\\
Volume&Roman type\\
Page numbers&Roman type\\
\br
\end{tabular}
\end{table}

Points to note are:
\medskip
\begin{itemize}
\item Book titles are in italic and should be spelt out in full with initial capital letters for all except minor words. Words such as Proceedings, Symposium, International, Conference, Second, etc should be abbreviated to {\it Proc.}, {\it Symp.}, {\it Int.}, {\it Conf.}, {\it 2nd}, respectively, but the rest of the title should be given in full, followed by the date of the conference and the town or city where the conference was held. For Laboratory Reports the Laboratory should be spelt out wherever possible, e.g. {\it Argonne National Laboratory Report}.
\item The volume number, for example vol 2, should be followed by the editors, if any, in a form such as `ed A J Smith and P R Jones'. Use {\it et al} if there are more than two editors. Next comes the town of publication and publisher, within brackets and separated by a colon, and finally the page numbers preceded by p if only one number is given or pp if both the initial and final numbers are given.
\end{itemize}

Examples taken from published papers:
\medskip

\numrefs{99}
\item Kurata M 1982 {\it Numerical Analysis for Semiconductor Devices} (Lexington, MA: Heath)
\item Selberherr S 1984 {\it Analysis and Simulation of Semiconductor Devices} (Berlin: Springer)
\item Sze S M 1969 {\it Physics of Semiconductor Devices} (New York: Wiley-Interscience)
\item Dorman L I 1975 {\it Variations of Galactic Cosmic Rays} (Moscow: Moscow State University Press) p 103
\item Caplar R and Kulisic P 1973 {\it Proc. Int. Conf. on Nuclear Physics (Munich)} vol 1 (Amsterdam: 	North-Holland/American Elsevier) p 517
\item Cheng G X 2001 {\it Raman and Brillouin Scattering-Principles and Applications} (Beijing: Scientific) 
\item Szytula A and Leciejewicz J 1989 {\it Handbook on the Physics and Chemistry of Rare Earths} vol 12, ed K A Gschneidner Jr and L Erwin (Amsterdam: Elsevier) p 133
\item Kuhn T 1998 {\it Density matrix theory of coherent ultrafast dynamics Theory of Transport Properties of Semiconductor Nanostructures} (Electronic Materials vol 4) ed E Sch\"oll (London: Chapman and Hall) chapter 6 pp 173--214
\endnumrefs

\section{Tables and table captions}
Tables should be numbered serially and referred to in the text 
by number (table 1, etc, {\bf rather than} tab. 1). Each table should be a float and be positioned within the text at the most convenient place near to where it is first mentioned in the text. It should have an 
explanatory caption which should be as concise as possible. 

\subsection{The basic table format}
The standard form for a table is:
\begin{verbatim}
\begin{table}
\caption{\label{label}Table caption.}
\begin{center}
\begin{tabular}{llll}
\br
Head 1&Head 2&Head 3&Head 4\\
\mr
1.1&1.2&1.3&1.4\\
2.1&2.2&2.3&2.4\\
\br
\end{tabular}
\end{center}
\end{table}
\end{verbatim}

The above code produces table~\ref{ex}.

\begin{table}[h]
\caption{\label{ex}Table caption.}
\begin{center}
\begin{tabular}{llll}
\br
Head 1&Head 2&Head 3&Head 4\\
\mr
1.1&1.2&1.3&1.4\\
2.1&2.2&2.3&2.4\\
\br
\end{tabular}
\end{center}
\end{table}

Points to note are:
\medskip
\begin{enumerate}
\item The caption comes before the table.
\item The normal style is for tables to be centred in the same way as
equations. This is accomplished
by using \verb"\begin{center}" \dots\ \verb"\end{center}".

\item The default alignment of columns should be aligned left.

\item Tables should have only horizontal rules and no vertical ones. The rules at
the top and bottom are thicker than internal rules and are set with
\verb"\br" (bold rule). 
The rule separating the headings from the entries is set with
\verb"\mr" (medium rule). These commands do not need a following double backslash.

\item Numbers in columns should be aligned as appropriate, usually on the decimal point;
to help do this a control sequence \verb"\lineup" has been defined 
which sets \verb"\0" equal to a space the size of a digit, \verb"\m"
to be a space the width of a minus sign, and \verb"\-" to be a left
overlapping minus sign. \verb"\-" is for use in text mode while the other
two commands may be used in maths or text.
(\verb"\lineup" should only be used within a table
environment after the caption so that \verb"\-" has its normal meaning
elsewhere.) See table~\ref{tabone} for an example of a table where
\verb"\lineup" has been used.
\end{enumerate}

\begin{table}[h]
\caption{\label{tabone}A simple example produced using the standard table commands 
and $\backslash${\tt lineup} to assist in aligning columns on the 
decimal point. The width of the 
table and rules is set automatically by the 
preamble.} 

\begin{center}
\lineup
\begin{tabular}{*{7}{l}}
\br                              
$\0\0A$&$B$&$C$&\m$D$&\m$E$&$F$&$\0G$\cr 
\mr
\0\023.5&60  &0.53&$-20.2$&$-0.22$ &\01.7&\014.5\cr
\0\039.7&\-60&0.74&$-51.9$&$-0.208$&47.2 &146\cr 
\0123.7 &\00 &0.75&$-57.2$&\m---   &---  &---\cr 
3241.56 &60  &0.60&$-48.1$&$-0.29$ &41   &\015\cr 
\br
\end{tabular}
\end{center}
\end{table}
 
\section{Figures and figure captions}
Figures must be included in the source code of an article at the appropriate place in the text not grouped together at the end. 

Each figure should have a brief caption describing it and, if 
necessary, interpreting the various lines and symbols on the figure. 
As much lettering as possible should be removed from the figure itself and 
included in the caption. If a figure has parts, these should be 
labelled ($a$), ($b$), ($c$), etc. 
\Tref{blobs} gives the definitions for describing symbols and lines often
used within figure captions (more symbols are available
when using the optional packages loading the AMS extension fonts).

\begin{table}[h]
\caption{\label{blobs}Control sequences to describe lines and symbols in figure 
captions.}
\begin{center}
\begin{tabular}{lllll}
\br
Control sequence&Output&&Control sequence&Output\\
\mr
\verb"\dotted"&\dotted        &&\verb"\opencircle"&\opencircle\\
\verb"\dashed"&\dashed        &&\verb"\opentriangle"&\opentriangle\\
\verb"\broken"&\broken&&\verb"\opentriangledown"&\opentriangledown\\
\verb"\longbroken"&\longbroken&&\verb"\fullsquare"&\fullsquare\\
\verb"\chain"&\chain          &&\verb"\opensquare"&\opensquare\\
\verb"\dashddot"&\dashddot    &&\verb"\fullcircle"&\fullcircle\\
\verb"\full"&\full            &&\verb"\opendiamond"&\opendiamond\\
\br
\end{tabular}
\end{center}
\end{table}


Authors should try and use the space allocated to them as economically as possible. At times it may be convenient to put two figures side by side or the caption at the side of a figure. To put figures side by side, within a figure environment, put each figure and its caption into a minipage with an appropriate width (e.g. 3in or 18pc if the figures are of equal size) and then separate the figures slightly by adding some horizontal space between the two minipages (e.g. \verb"\hspace{.2in}" or \verb"\hspace{1.5pc}". To get the caption at the side of the figure add the small horizontal space after the \verb"\includegraphics" command and then put the \verb"\caption" within a minipage of the appropriate width aligned bottom, i.e. \verb"\begin{minipage}[b]{3in}" etc (see code in this file used to generate figures 1--3).

Note that it may be necessary to adjust the size of the figures (using optional arguments to \verb"\includegraphics", for instance \verb"[width=3in]") to get you article to fit within your page allowance or to obtain good page breaks.

\begin{figure}[h]
\begin{minipage}{14pc}
\includegraphics[width=14pc]{name.eps}
\caption{\label{label}Figure caption for first of two sided figures.}
\end{minipage}\hspace{2pc}%
\begin{minipage}{14pc}
\includegraphics[width=14pc]{name.eps}
\caption{\label{label}Figure caption for second of two sided figures.}
\end{minipage} 
\end{figure}

\begin{figure}[h]
\includegraphics[width=14pc]{name.eps}\hspace{2pc}%
\begin{minipage}[b]{14pc}\caption{\label{label}Figure caption for a narrow figure where the caption is put at the side of the figure.}
\end{minipage}
\end{figure}

Using the graphicx package figures can be included using code such as:
\begin{verbatim}
\begin{figure}
\begin{center}
\includegraphics{file.eps}
\end{center}
\caption{\label{label}Figure caption}
\end{figure}
\end{verbatim}

\section*{References}
\begin{thebibliography}{9}
\bibitem{iopartnum} IOP Publishing is to grateful Mark A Caprio, Center for Theoretical Physics, Yale University, for permission to include the {\tt iopart-num} \BibTeX package (version 2.0, December 21, 2006) with  this documentation. Updates and new releases of {\tt iopart-num} can be found on \verb"www.ctan.org" (CTAN). 
\end{thebibliography}

\fi
\end{document}


